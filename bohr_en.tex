% arara: pdflatex: { shell: on }
% arara: pdflatex
\documentclass[toc=index]{cnpkgdoc}
\docsetup{
  pkg      = bohr ,
  cmd      = \BOHR ,
  code-box = {
    skipbelow        = .5\baselineskip plus .5ex minus .5ex ,
    skipabove        = .5\baselineskip plus .5ex minus .5ex ,
    roundcorner      = 3pt ,
    innerleftmargin  = 1.5em ,
    innerrightmargin = 1.5em
  } ,
  gobble   = 1
}
\addcmds{bohr,ch,setbohr}
\usepackage{chemmacros}

\usepackage[osf]{libertine}
\usepackage[supstfm=libertinesups]{superiors}
\usepackage{fnpct}

\usepackage{embrac}[2012/06/29]
\ChangeEmph{[}[,.02em]{]}[.055em,-.08em]
\ChangeEmph{(}[-.01em,.04em]{)}[.04em,-.05em]

\renewcommand*\othersectionlevelsformat[3]{\textcolor{main}{#3\autodot}\enskip}
\renewcommand*\partformat{\textcolor{main}{\partname~\thepart\autodot}}

\usepackage{filecontents}
\begin{filecontents}{\jobname.ist}
 heading_prefix "{\\bfseries "
 heading_suffix "\\hfil}\\nopagebreak\n"
 headings_flag  1
 delim_0 "\\dotfill "
 delim_1 "\\dotfill "
 delim_2 "\\dotfill "
 delim_r "\\textendash"
 suffix_2p "\\nohyperpage{\\,f.}"
 suffix_3p "\\nohyperpage{\\,ff.}"
\end{filecontents}

\usepackage{imakeidx}
% \indexsetup{noclearpage}
\makeindex[columns=2,intoc,options={-sl \jobname.ist}]

\TitlePicture{%
  \parbox{.7\linewidth}{This package provides means for the creation of simple
  Bohr models of atoms up to the atomic number 54. It is inspired by a question
  on \url{http://tex.stackexchange.com/}: \href{http://tex.stackexchange.com/%
  questions/73410/draw-bohr-atomic-model-with-electron-shells-in-tex}%
  {Draw Bohr atomic model with electron shells in TeX?}}%
}

\newcommand*\Default[1]{%
  \hfill\llap
    {%
      \ifblank{#1}%
        {(initially~empty)}%
        {Default:~\code{#1}}%
    }%
  \newline
}
\newcommand*\required{\hfill\llap{required}\newline}
\newcommand*\optional{\hfill\llap{optional}\newline}

\begin{document}

\section{Licence and Requirements}
\BOHR is placed under the terms of the LaTeX Project Public License,
version 1.3 or later (\url{http://www.latex-project.org/lppl.txt}).
It has the status ``maintained.''

\BOHR loads and needs the packages \paket{tikz} and \paket{etoolbox}.

\section{Usage}\secidx{Usage}
\BOHR is used like any other \LaTeXe\ package:
\begin{beispiel}[code only]
 \usepackage{bohr}
\end{beispiel}
It has no options so that's it.

\BOHR provides a single command, \cmd{bohr}, which creates the models:
\begin{beschreibung}
 \Befehl{bohr}\oa{<num of shells>}\ma{<number of electrons>}\ma{<atom name>}
\end{beschreibung}
This is described best by an example:
\begin{beispiel}
 \bohr{3}{Li}
\end{beispiel}
There is not much more to it. Another example using the optional argument:
\begin{beispiel}
 \bohr[2]{2}{$\mathrm{Li^+}$}
\end{beispiel}
\secidx*{Usage}

\section{Customization}\secidx{Customization}
\BOHR provides a few options to customize the appearance:
\begin{beschreibung}
 \Befehl{setbohr}\newline
   Options are set in a key/value syntax using this command.
 \Option{atom-style}{<code>}\Default{}
   This code will be placed immediatly before the third argument of \cmd{bohr}.
   The last macro in it may need one argument.
 \Option{name-options-set}{<tikz>}\Default{}
   This value is passed to the options of the \cmd*{node} the third argument of
   \cmd{bohr} is placed in.
 \Option{name-options-add}{<tikz>}\Default{}
   This value will be added to options set with \key{name-options-set}.
 \Option{nucleus-option-set}{<tikz>}\Default{draw=black!80,fill=black!10,opacity=.25}
   This value is passed to the options of the \cmd*{draw} command that draws the
   circle around the name-node.
 \Option{nucleus-options-add}{<tikz>}\Default{}
   This value will be added to options set with \key{nucleus-options-set}.
 \Option{nucleus-radius}{<dim>}\Default{1em}
   The radius of the circle around the name-node.
 \Option{electron-options-set}{<tikz>}\Default{blue!50!black!50}
   This value is passed to the options of the \cmd*{fill} command that draws the
   electrons.
 \Option{electron-options-add}{<tikz>}\Default{}
   This value will be added to options set with \key{electron-options-set}.
 \Option{electron-radius}{<dim>}\Default{1.5pt}
   The radius of the circles that represent the electrons.
 \Option{shell-options-set}{<tikz>}\Default{draw=blue!75,thin}
   This value is passed to the options of the \cmd*{draw} command that draws the
   circles that represent the shells.
 \Option{shell-options-add}{<tikz>}\Default{}
   This value will be added to options set with \key{shell-options-set}.
 \Option{shell-dist}{<dim>}\Default{1em}
   The distance between the nucleus and the first shell and between subsequent
   shells.
\end{beschreibung}

\begin{beispiel}
 \setbohr{name-options-set={font=\footnotesize\sffamily}}
 \bohr{2}{He} \bohr{7}{N}
\end{beispiel}

\begin{beispiel}
 % uses package `chemmacros'
 \setbohr{atom-style={\footnotesize\sffamily\ch}}
 \bohr{0}{H+} \bohr{10}{F-}
\end{beispiel}

\begin{beispiel}
 \setbohr{
   shell-options-add = dashed,
   shell-dist        = .5em
 }
 \bohr{6}{C} \bohr{19}{K}
\end{beispiel}
\secidx*{Customization}

\section{Implementation}
\implementation

\printindex
\end{document}